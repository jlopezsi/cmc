
\documentclass[preprint,times,authoryear,12pt]{elsarticle}

%% The amssymb package provides various useful mathematical symbols
\usepackage{amssymb,amsmath}


%%%%%%%%%% Remove the following before submission %%%%%%%%%%%%%%%%%%

\usepackage{mathspec,xltxtra,xunicode}
%\usepackage{unicode-math}
\defaultfontfeatures{Scale=MatchLowercase}
%\setmainfont[Mapping=tex-text,Numbers=OldStyle]{Palatino LT Std}
%\setmainfont[Ligatures=TeX,Numbers=OldStyle]{Minion Pro}
%\setsansfont[Mapping=tex-text]{ITC Legacy Sans Std Medium}
%\setmonofont{Bitstream Vera Sans Mono}
%\setmathfont(Digits,Latin,Greek)[Script=Math,Uppercase=Italic,Lowercase=Italic]{Minion Math Semibold}
%\setmathfont[range={\mathbfup->\mathup}]{MinionMath-Bold.otf}
%\setmathfont[range={\mathbfit->\mathit}]{MinionMath-Bold.otf}
%\setmathfont[range={\mathit->\mathit}]{MinionMath-Bold.otf}

%%%%%%%%%% Remove the above before submission %%%%%%%%%%%%%%%%%%

%% The amsthm package provides extended theorem environments
%% \usepackage{amsthm}

%% The lineno packages adds line numbers. Start line numbering with
%% \begin{linenumbers}, end it with \end{linenumbers}. Or switch it on
%% for the whole article with \linenumbers after \end{frontmatter}.
\usepackage{lineno}
\usepackage{graphicx}
\usepackage{xspace}
\usepackage{bm}
\usepackage{longtable}
\usepackage{hyphenat}
\usepackage{lipsum}
\usepackage{url}
\usepackage{madsen-macros}

\journal{Unnamed Journal}

% Pandoc toggle for numbering sections (defaults to be off)
\setcounter{secnumdepth}{0}

% Pandoc header


\begin{document}

\begin{frontmatter}


\title{RESEARCH NOTE:\\ Combinatorial Structure of the Deterministic Seriation Method with Multiple Spatial Solutions}

\author{Mark E. Madsen}
\address{Department of Anthropology, Box 353100, University of Washington, Seattle WA, 98195 USA}
\ead{mark@madsenlab.org}
\ead[url]{http://madsenlab.org}

\author{Carl P. Lipo}
\address{Department of Anthropology and IIRMES, 1250 Bellflower Blvd, California State University at Long Beach, Long Beach CA, 90840 USA}
\ead{Carl.Lipo@csulb.edu}
\ead[url]{http://lipolab.org}

\end{frontmatter}

\subparagraph{keywords}\label{keywords}

seriation \textbar{} combinatorics \textbar{} algorithms \textbar{}
cultural transmission

\section{Abstract}\label{abstract}

Lorem ipsum dolor sit amet, consectetur adipiscing elit. Sed dictum
mauris quis turpis pellentesque, quis porta arcu varius. Suspendisse
dictum at odio id luctus. Curabitur ut tristique nisl. Pellentesque
habitant morbi tristique senectus et netus et malesuada fames ac turpis
egestas. Vivamus molestie at elit a sodales. Vestibulum suscipit leo sed
semper facilisis. Nam lacinia lobortis imperdiet. Nulla facilisi. Duis
suscipit elementum risus, vitae molestie eros accumsan ut. Nulla
facilisi. Praesent quis adipiscing sem. Interdum et malesuada fames ac
ante ipsum primis in faucibus.

\section{Introduction}\label{introduction}

Lorem ipsum dolor sit amet, consectetur adipiscing elit. Sed dictum
mauris quis turpis pellentesque, quis porta arcu varius. Suspendisse
dictum at odio id luctus. Curabitur ut tristique nisl. Pellentesque
habitant morbi tristique senectus et netus et malesuada fames ac turpis
egestas. Vivamus molestie at elit a sodales. Vestibulum suscipit leo sed
semper facilisis. Nam lacinia lobortis imperdiet. Nulla facilisi. Duis
suscipit elementum risus, vitae molestie eros accumsan ut. Nulla
facilisi. Praesent quis adipiscing sem. Interdum et malesuada fames ac
ante ipsum primis in faucibus.

\section{Single Seriation
Combinatorics}\label{single-seriation-combinatorics}

Since the factorial function grows so quickly, the computational cost of
determining the correct permutation within a given seriation solution
group is controlled by the size of the largest subset, especially if the
other subsets are relatively small, as in the previous example. At
worst, for a solution set with $m$ solution groups, $m-1$ solution
groups will contain 1 assemblage each, and the last solution group will
consist of the remaining $n-m-1$ assemblages. This means, of course,
that the worst case would involve consideration of on the order of
$(n-m-1)!$ permutations within each solution group, for each of the
subsets given by Equation \ref{eq:sum-stirling}. This yields:

\[\sum_{m=1}^n \stirlingsubset{n}{m} (n-m-1)!\]

Table \ref{tab:total-mult} gives the total number of possible solutions
for assemblages ranging from 4 to 100, where solutions may fall into
multiple seriation groups of any size.

\begin{verbatim}
Error in `[.data.frame`(tli, 1:4, 1:10): undefined columns selected
\end{verbatim}

\begin{verbatim}
Error in print(xt3, include.rownames = FALSE): object 'xt3' not found
\end{verbatim}

\section{Conclusions}\label{conclusions}

Lorem ipsum dolor sit amet, consectetur adipiscing elit. Sed dictum
mauris quis turpis pellentesque, quis porta arcu varius. Suspendisse
dictum at odio id luctus. Curabitur ut tristique nisl. Pellentesque
habitant morbi tristique senectus et netus et malesuada fames ac turpis
egestas. Vivamus molestie at elit a sodales. Vestibulum suscipit leo sed
semper facilisis. Nam lacinia lobortis imperdiet. Nulla facilisi. Duis
suscipit elementum risus, vitae molestie eros accumsan ut. Nulla
facilisi. Praesent quis adipiscing sem. Interdum et malesuada fames ac
ante ipsum primis in faucibus.

\section{Acknowledgements}\label{acknowledgements}

Lorem ipsum dolor sit amet, consectetur adipiscing elit. Sed dictum
mauris quis turpis pellentesque, quis porta arcu varius. Suspendisse
dictum at odio id luctus. Curabitur ut tristique nisl. Pellentesque
habitant morbi tristique senectus et netus et malesuada fames ac turpis
egestas. Vivamus molestie at elit a sodales. Vestibulum suscipit leo sed
semper facilisis. Nam lacinia lobortis imperdiet. Nulla facilisi. Duis
suscipit elementum risus, vitae molestie eros accumsan ut. Nulla
facilisi. Praesent quis adipiscing sem. Interdum et malesuada fames ac
ante ipsum primis in faucibus.


%% References with bibTeX database:

\bibliographystyle{model2-names}
\bibliography{seriationcombinatorics}









\end{document}

%%
%% End of file `elsarticle-template-2-harv.tex'.
